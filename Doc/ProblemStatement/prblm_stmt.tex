\documentclass{article}

\usepackage{tabularx}
\usepackage{booktabs}
\usepackage{xcolor}
\usepackage[normalem]{ulem}

\title{SE 3XA3: Problem Statement\\ Scrabble}

\author{Team \#214, The Trifecta
		\\ Kanakabha Choudhri, choudhrk
		\\ Raymond Tu, tur1
		\\ Lucia Cristiano, cristial
}

\date{January 24th 2020}

%\input{../Comments}
%\sout{}

\begin{document}

\begin{table}[hp]
\caption{Revision History} \label{TblRevisionHistory}
\begin{tabularx}{\textwidth}{llX}
\toprule
\textbf{Date} & \textbf{Developer(s)} & \textbf{Change}\\
\midrule
Jan 22, 2020 & Kanakabha, Lucia, Raymond & creating problem statement\\
\textcolor{red}{Mar 27, 2020} & \textcolor{red}{Lucia} & \textcolor{red}{Adding revision 1 edits}\\
\bottomrule
\end{tabularx}
\end{table}

\newpage

\maketitle

\section*{Problem Statement}
\subsection*{What problem are you trying to solve?}

As Team Trifecta we have noticed that there is a lack of accessible and well-designed classic board games available to play digitally. The current version of this Scrabble game is not easily usable for those individuals who are unfamiliar with using the command line terminal and is not aesthetically pleasing. Our aim is to create a version of the board game Scrabble that can be played by people on a computer using a graphical user interface(GUI) to provide a straightforward way of playing and allow for fun and friendly competition between users.\\

\subsection*{What is the context of the problem you are solving?}

The Scrabble project can be played by two to four people on a single desktop or laptop computer that has the ability to compile and run python 3 files. The users will only have to interact with the interface once the game has been launched. This project would fit into the category of gaming and entertainment since it is a game played by users for amusement and a way to pass time. The game can also be seen as educational as players can expand their English vocabulary by participating in the game. \\

\subsection*{Who are the stakeholders and why is this important?}

The Scrabble project has several persons, or stakeholders, that can impact the course of the project. One group of stakeholders are SFWRENG 3XA3 supervisors, Dr. Asghar Bokhari and teaching assistant Andrew Lucentini. These individuals have a vested interest in the project as their feedback directly impacts the requirements and development of the project. Another group of stakeholders are the original author of the repository, fayrose, and other Github users who have forked the project. They have a desire to see the project succeed and improve, these features can then either be applied to their own projects or improved upon. \textcolor{red}{Another group of stakeholders for the project are the users of the Scrabble game. These individuals are considered stakeholders, as their experience and enjoyment of the game during testing can result in significant changes to the product, as well as validation that the product meets its requirements. These findings will shape the final product.} Finally, the players of the game are also stakeholders as the project will be shaped by their feedback about how usable and accessible the game is during gameplay.

\end{document}
